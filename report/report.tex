% arara: pdflatex
% arara: bibtex
% arara: pdflatex
% arara: pdflatex
\documentclass[10pt, conference, letterpaper]{IEEEtran}

%\usepackage{algorithm}
%\usepackage{algorithmicx}
%\usepackage{algpseudocode}
\usepackage[utf8]{inputenc}
\usepackage{amsfonts}
\usepackage{amsmath}
\usepackage{amssymb}
\usepackage[T1]{fontenc}
\usepackage{microtype}
%\usepackage{mathtools}
\usepackage[dvipsnames]{xcolor}
\usepackage{graphicx}
\usepackage{caption}
\usepackage{subcaption}
\usepackage{booktabs}
%\usepackage{import}
\usepackage{multirow}
\usepackage{cite}
\usepackage[export]{adjustbox}
\usepackage{breqn}
%\usepackage{mathrsfs}
\usepackage{acronym}
\usepackage[keeplastbox]{flushend}
\usepackage{setspace}
\usepackage{bm}
\usepackage{stackengine}

\usepackage{smartdiagram}
\usepackage{tikz}
\usesmartdiagramlibrary{additions}

\usepackage{hyperref}
\definecolor{azure}     {rgb}{0,0.5,1}
\definecolor{dkpowder}  {rgb}{0,0.2,0.7}
\definecolor{deepred}   {rgb}{0.7,0,0}
\definecolor{deepblue}  {rgb}{0,0,0.7}
\definecolor{deepgreen} {rgb}{0,0.5,0}
\definecolor{deeporange}{rgb}{0.91, 0.41, 0.17}
\hypersetup{%
    pdfpagemode  = {UseOutlines},
    bookmarksopen,
    pdfstartview = {FitH},
    colorlinks,
    linkcolor = {dkpowder},
    citecolor = {dkpowder},
    urlcolor  = {dkpowder},
}

%\addto\extrasenglish{%
%  \renewcommand{\sectionautorefname}{Section}%
%  \renewcommand{\subsectionautorefname}{Subsection}%
%}

\usepackage{listings}
%\lstset{%
%backgroundcolor=\color[gray]{.85},
%basicstyle=\small\ttfamily,
%breaklines = true,
%keywordstyle=\color{red!75},
%columns=fullflexible,
%}%

\lstdefinelanguage{BibTeX}{%
keywords={%
    @article,@book,@collectedbook,@conference,@electronic,@ieeetranbstctl,%
    @inbook,@incollectedbook,@incollection,@injournal,@inproceedings,%
    @manual,@mastersthesis,@misc,@patent,@periodical,@phdthesis,@preamble,%
    @proceedings,@standard,@string,@techreport,@unpublished%
},
comment=[l][\itshape]{@comment},
sensitive=false,
}

% listings settings from classicthesis package by Andr\'{e} Miede
%\lstset{language=[LaTeX]Tex,%C++,
%    keywordstyle=\color{RoyalBlue},%\bfseries,
%    basicstyle=\small\ttfamily,
%    %identifierstyle=\color{NavyBlue},
%    commentstyle=\color{Green}\ttfamily,
%    stringstyle=\rmfamily,
%    numbers=none,%left,%
%    numberstyle=\scriptsize,%\tiny
%    stepnumber=5,
%    numbersep=8pt,
%    showstringspaces=false,
%    breaklines=true,
%    frameround=ftff,
%    frame=single
%    %frame=L
%}

\lstset{%
    language = Python,
    frame   = lines, % shapes
    %numbers = left,
    breaklines = true,
    captionpos = b,
    showstringspaces  = false,
    breakatwhitespace = true,
    escapeinside={<@}{@>}, % to use and evaluate latex commands inside code
    %abovecaptionskip  = \bigskipamount,
    %
    basicstyle   = \scriptsize\ttfamily,
    numberstyle  = \scriptsize,
    emphstyle    = \color{azure},
    stringstyle  = \color{deepred},
    keywordstyle = \color{deepblue},
    commentstyle = \color{deepgreen},
    %
    morekeywords = {as,class,__init__,self,False,True},
    %emph = {Sequential},
    emph = {Dense,Conv1D,Conv2D,Activation,Input,Flatten,BatchNormalization,Dropout,MaxPooling1D,MaxPooling2D,LSTM,GRU,TimeDistributed}
}

%\lstset{literate=%
%    *{0}{{{\color{red}0}}}1
%    {1}{{{\color{red}1}}}1
%    {2}{{{\color{red}2}}}1
%    {3}{{{\color{red}3}}}1
%    {4}{{{\color{red}4}}}1
%    {5}{{{\color{red}5}}}1
%    {6}{{{\color{red}6}}}1
%    {7}{{{\color{red}7}}}1
%    {8}{{{\color{red}8}}}1
%    {9}{{{\color{red}9}}}1
%}

\renewcommand{\thetable}{\arabic{table}}
\renewcommand{\thesubtable}{\alph{subtable}}

\DeclareMathOperator*{\argmin}{arg\,min}
\DeclareMathOperator*{\argmax}{arg\,max}

\def\delequal{\mathrel{\ensurestackMath{\stackon[1pt]{=}{\scriptscriptstyle\Delta}}}}

\graphicspath{{images/}}
\setlength{\belowcaptionskip}{0mm}
\setlength{\textfloatsep}{7pt}

\newcommand{\eq}[1]{Eq.~\eqref{#1}}
\newcommand{\fig}[1]{Fig.~\ref{#1}}
\newcommand{\tab}[1]{Tab.~\ref{#1}}
\newcommand{\secref}[1]{Section~\ref{#1}}

\newcommand\MR[1]{\textcolor{blue}{#1}}
\newcommand\red[1]{\textcolor{red}{#1}}
\newcommand\comment[1]{\textcolor{ForestGreen}{[#1]}}

%\renewcommand{\baselinestretch}{0.98}
%\renewcommand{\bottomfraction}{0.8}
%\setlength{\abovecaptionskip}{0pt}
\setlength{\columnsep}{0.1in}

% fix for when no cite is in document but want bib to be shown
\makeatletter
\def\endthebibliography{%
  \def\@noitemerr{\@latex@warning{Empty `thebibliography' environment}}%
  \endlist}
\makeatother

% label colors
\definecolor{label-running} {RGB}{ 31,119,180}
\definecolor{label-walking} {RGB}{255,127, 14}
\definecolor{label-jumping} {RGB}{ 44,160, 44}
\definecolor{label-standing}{RGB}{148,103,189}
\definecolor{label-sitting} {RGB}{140, 86, 75}
\definecolor{label-lying}   {RGB}{127,127,127}
\definecolor{label-falling} {RGB}{188,189, 34}
\definecolor{label-transit} {RGB}{ 23,190,207}

% pass the color to draw the box
\newcommand{\labelbox}[1]{%
    \raisebox{1.6pt}{\fcolorbox{black}{#1}{\rule[2.5pt]{5pt}{0pt}}}
}

% \IEEEoverridecommandlockouts\IEEEpubid{\makebox[\columnwidth]{PUT COPYRIGHT NOTICE HERE \hfill} \hspace{\columnsep}\makebox[\columnwidth]{ }}
\frenchspacing
\IEEEoverridecommandlockouts

\title{Going deep into Human Activity Recognition}

\author{Elia Bonetto, Filippo Rigotto
\thanks{All authors are with the Department of Information Engineering, University of Padova, Italy. Email: \{bonettoe, rigottof\}@dei.unipd.it}}
%\author{Elia Bonetto$^\dag$, Filippo Rigotto$^\ddag$
%\thanks{$^\dag$Department of Information Engineering, University of Padova, Italy, email: bonettoe@dei.unipd.it}
%\thanks{$^\ddag$Department of Information Engineering, University of Padova, Italy, email: rigottof@dei.unipd.it}
%\thanks{Special thanks / acknowledgement go here.}
%}

\markboth{Bonetto \& Rigotto: Going deep into Human Activity Recognition}{Human Data Analytics, Spring 2019}

\begin{document}

\maketitle
%\thispagestyle{plain} % forced page numbers. TO DELETE before delivery
%\pagestyle{plain} % forced page numbers. TO DELETE before delivery

\begin{abstract}
In latest years, thanks to the increased number of smartphones and wearable devices integrating IMUs, Human Activity Recognition (HAR) has become a key research topic in monitored and assisted living either for medical or tracking reasons.
First attempts provided manual feature crafting, followed by analysis done either with deep neural networks or other approaches like Hidden Markov models.
More recently instead, direct analysis on raw signals has been attempted.
Here we continue this trend by exploring some possible approaches with convolutional and recurrent neural networks and look over automatic feature extraction techniques, such as autoencoders.
Most of the datasets in this fields are highly imbalanced and some classes lack of enough data.
To face this, we propose two augmentation techniques for rebalancing.
Finally, we introduce new ways and metrics to select the best learning epoch to address overfitting and get the best learning results overall.
Our tests confirm that augmenting the initial dataset is worth the effort, and we achieve performance that surpass what is declared for it.
Moreover, we discovered that working with raw signals in the sensor reference frame is better than working with their trasformation to the body frame.
As for encoded data by the means of our autoencoders, we could not find any performance improvement: in some cases, worse results are obtained.
\end{abstract}

\begin{IEEEkeywords}
Activity recognition, inertial sensors, machine learning, neural networks, autoencoders, deep classification
%\MR{A list of keywords defining the tools and the scenario. I would not go beyond {\it six} keywords.}
\end{IEEEkeywords}

% !TEX root = template.tex

\section{Introduction}
\label{sec:introduction}

\red{Maximum length for the whole report is 9 pages. Abstract, introduction and related work should take max two pages.}\\

\MR{A good way of structuring the introduction is as follows: 
\begin{itemize}
\item \textbf{General (short) intro:} One paragraph to introduce your work, describing the scenario {\it at large}, its relevance, to prepare the reader to what follows and convince her/him that the paper focuses on an important setup / problem. Please keep this part short (I usually do five to eight lines), as this part is rather standard, \textbf{but} at the same time it has to be there. 
\item \textbf{Put the problem into perspective:} A second paragraph where you immediately delve into the specific problem that you are tackling, starting to point the finger towards your contribution. Here, you describe the importance of such problem, providing examples (citing papers from the literature, possibly recent ones) of previous solutions attempts, and of why these failed {\it to provide a complete answer}. This second paragraph should not be too long, as otherwise the reader will get bored and will abandon your paper... It should be concisely written, something like 5 to 10 lines.
\item \textbf{Present the paper contribution:} A third paragraph were you state what you do in the paper, this should also be concisely written. A good rule of thumb is to make it max 10/15 lines. Here, you should state up front:
\begin{enumerate}
\item the problem at stake, 
\item the relevance of what you propose, 
\item the technique you use, possibly underlying its novelty, efficiency, 
\item underline the value/novelty of your proposal, referencing (recent) papers from the literature,
\item tell the reader how she/he can take advantage of your work, e.g., how your work/results can be reused/exploited to achieve further scientific, technical or practical goals.
 \end{enumerate}
\item \textbf{Summary of contributions:} After this, you may want to provide an itemized list to summarize the paper contributions. Rule of thumb: from three to six items, from three to four lines each.
\item \textbf{Closing (paper structure):} You finish up by detailing the paper structure, this should be three to four lines. It is customary to do so, although I admit it may be of little use. It usually goes like: {\it ``This report (paper) is structured as follows. In Section II we describe the state of the art, the system and data models are respectively expounded in Sections~III and~IV. The proposed signal processing technique is presented in Section~V and its performance evaluation is carried out in Section~VI.''}\\
\end{itemize}}

\MR{Lately, I tend to write introduction plus abstract within a single page. This forces me to focus on the important messages that I want to deliver about the paper, leaving out all the blah blah. \textbf{Remember:} 1) {\it less is more}, 2) writing a compact ({\it snappy}) piece of technical text is much more difficult than writing with no space constraints.}
% !TEX root = template.tex

\section{Related Work}
\label{sec:related_work}

\MR{\begin{itemize}
\item \textbf{Goal:} The goal of this section is to describe what has been done so far in {\it the} literature. You should focus on and briefly describe the work done in the best papers that you have read.
\item \textbf{Length:} One full column is fine but often this takes one column and a half. It is very easy to use a full page, although this is because you are sloppy... if you carefully go through the one page long version you often find it is possible to compact it in one column and a half. In any event, I would make this section no longer than one page, this leads to an overall {\it two pages} including abstract, introduction and related work. I believe this is a fair amount of space in most cases.
\item \textbf{Approach:} For each you should comment on the paper's contribution, on the good and important findings of such paper and also, 1) on why these findings are not enough and 2) how these findings are improved upon/extended by the work that you present here. At the end of the section, you may recap the main paper contributions (maybe one or two, the most important ones) and how these extend/improve upon previous work.
\item \textbf{References:} please follow this {\it religiously}. It will help you a lot. Use the Latex \texttt{Bibtex} tool to manage the bibliography. A \texttt{Bibtex} example file, named \texttt{biblio.bib} is provided with this template.\\
\end{itemize}}
%
\MR{Note that some of the above fields may not be shown when you compile the Latex file, but this depends on the bibliography settings (dictated by the specific Latex style that you load at the beginning of the document). You may decide to include additional pieces of information in a given bibliographic entry, but please, \textbf{be consistent} across all the entries, i.e., use the same fields for the same publication type. Exceptions are in the (rare) cases where some of the fields are not available (e.g., the paper {\it volume}, {\it number} or the {\it pages}).}

% !TEX root = report.tex

\section{Processing pipeline}\label{sec:processing_pipeline}

\begin{figure*}
    \centering
        \smartdiagramset{
        border color=none,
        uniform color list=CornflowerBlue!60 for 6 items,
        back arrow disabled=true,
        additions={additional item offset=4mm,
            additional item shadow=drop shadow,
            additional item bottom color=CornflowerBlue!60,
            additional arrow color=CornflowerBlue!60,
            }
        }
        \smartdiagramadd[flow diagram:horizontal]{Raw signals,Preprocessing,Models,Parameters definition,Training,Evaluation}{below of module3/AE, below of module5/SVM/LR}
        \smartdiagramconnect{->}{additional-module1/module3, additional-module1/additional-module2}
        \begin{tikzpicture}[remember picture,overlay]% modified from p. 47 of manual
            \draw[additional item arrow type] (module2) |- ([yshift=-9mm]module2.south) -- (additional-module1);
            \draw[additional item arrow type] (additional-module1) |- (additional-module1.north) -- (module3);
            \draw[additional item arrow type] (additional-module2) |- ([xshift=17.5mm]additional-module2.east) -- (module6.south);
        \end{tikzpicture}\vspace{14mm}
    \caption{Full processing pipeline from raw signals' dataset to networks evaluation.}
    \label{fig:full_proc}
\end{figure*}

As mentioned in the introduction, this work uses sensor data collected by the Institute of Communications and Navigation, German Aerospace Center (DLR)\footnote{\url{https://www.dlr.de/kn/desktopdefault.aspx/tabid-12705/22182_read-50785}}, base of the work done in~\cite{FrankNadales}.
Nonetheless, source code is modular, to enable the possibility to use other datasets, like the well known from UCI\footnote{\url{https://archive.ics.uci.edu/ml/datasets/human+activity+recognition+using+smartphones}}.
The processing phase of raw signals data, needed to build an usable dataset to feed the networks, is kept at a minimum without using for example filters or noise reduction techniques to avoid distorting too much the dataset, to further prove the usability of non-elaborated features. In the original version, the dataset is not balanced w.r.t. the classes frequency.
To contrast this, after the generation of training and test sets, augmentation is experimented and used to create two further dataset versions: using a hand-crafted method consisting of random data rotation and shuffling, and an algorithm suited for this task such as ADASYN~\cite{He-ADASYN}.

In the first part of this work, classification of data is performed using common architectures inspired from the literature: 1D and 2D CNNs, LSTM and GRU-based RNNs.

A second part of the study involves the generation of features from data thanks both to a stack of convolutional layers and to autoencoders, made of convolutional, LSTM or both kind of layers. Classification is then performed by means of the previous defined networks, generating a mixture of possible configurations, and of other classification algorithms like SVM and logistic regression (LR), using a stochastic gradient descent (SGD) optimization method.

To select the best training epoch we have introduced some brand-new metrics to counteract eventual overfitting problems, variability of the training progress and to take into account not only accuracy but also precision and recall scores.
Validation split was not carried out due to the small dataset size and the fact that augmentation only occurs on the training set: validating over an augmented dataset could have led to overfitting the data, resulting in poor performance.

A visual representation of the full process is in \fig{fig:full_proc}.

\section{Signals preprocessing}\label{sec:model}

\begin{figure}
\begin{center}
    \smartdiagramset{
        border color=none,
        uniform color list=CornflowerBlue!60 for 6 items,
        back arrow disabled=true,
        additions={additional item offset=8mm,
            additional item shadow=drop shadow,
            additional item bottom color=CornflowerBlue!60,
            additional arrow color=CornflowerBlue!60,
        }
    }
    \smartdiagramadd[flow diagram:horizontal]{Raw signals, Flattening, Framing}{below of module1/Splitting, below of module2/Augmentation, below of module3/Normalization}
    \smartdiagramconnect{->}{additional-module1/additional-module2,additional-module2/additional-module3}
  \begin{tikzpicture}[remember picture,overlay]
    \draw[additional item arrow type] (module3) |- ([yshift=-3mm]module1.south) -- (additional-module1);
    \draw[additional item arrow type] (additional-module1) |- ([yshift=-4mm]additional-module3.south) -- (additional-module3);
  \end{tikzpicture}\vspace{22mm}

    \caption{Data preprocessing steps.}
    \label{fig:preprocessing}
\end{center}
\end{figure}

The dataset contains IMU sensor measurements of several scheduled movements and activities taken and labelled by different people.
The \textsc{xsens MTx} IMU device is positioned on the belt of the user, and collects data at a rate of 100Hz and each measure consists of the snapshot time plus raw readings of an accelerometer, a gyroscope and a magnetometer.

Each sensor outputs 3D data, according to the device reference frame, and also computes the necessary transformation matrix $\bm{T}$ of shape $3\times 3$ to change the reference frame into a body-aligned frame.
Mathematically, for each sensor data $\bm{v}_s$:
\[ \begin{bmatrix}x_b\\y_b\\z_b\end{bmatrix} = \bm{v}_b = \bm{T} \bm{v}_s = \bm{T} \begin{bmatrix}x_s\\y_s\\z_s\end{bmatrix} \]

The original 17 tracked activities are grouped and reduced to 8 classes, mislabeling in classification are fixed.
The composition of samples in the dataset after relabeling is reported in the second column of \tab{tab:act_times}.

\begin{table}[ht]
    \centering
    \caption{Minutes for each activity in several versions of the dataset: the original, after framing and after augmentation.}
    \label{tab:act_times}
    \begin{tabular}{lccc}\toprule
                        & \multicolumn{3}{c}{Time} \\
        \qquad Activity & orig. & framing & augm. \\\midrule
        \labelbox{label-running} running    &  15 &  30 &  68 \\
        \labelbox{label-walking} walking    &  72 & 144 & 144 \\
        \labelbox{label-jumping} jumping    &   8 &  15 &  64 \\
        \labelbox{label-standing} standing  & 121 & 241 & 241 \\
        \labelbox{label-sitting} sitting    &  59 & 117 & 117 \\
        \labelbox{label-lying} lying        &  28 &  56 &  76 \\
        \labelbox{label-falling} falling    &   2 &   4 &  60 \\
        \labelbox{label-transit} transition &  60 & --- & --- \\
        \qquad\quad total                   & 365 & 609 & 772 \\\bottomrule
    \end{tabular}
\end{table}

Transitions are pruned from the dataset, and the rest of the data is sorted by activity and then framed in windows of length 128 and an overlapping rate of 50\%. This also implies padding sensor data with zeros to the nearest subsequent multiple of 64, to have entirely filled windows.
Updated times for each activity are in the third column of \tab{tab:act_times}.

The 70-30\% split of the dataset to obtain training and test sub-sets is done before the eventual augmentation of data, which is performed only on the training set: it is easy to notice from \tab{tab:act_times} that classes in the original dataset are not balanced.
Two different techniques have been applied to balance them:
\begin{itemize}
    \item using ADASYN~\cite{He-ADASYN} from \textit{imblearn}\footnote{\url{https://imbalanced-learn.readthedocs.io/en/stable/generated/imblearn.over_sampling.ADASYN.html}}: an adaptive algorithm that generate samples according to a density distribution, computed for every minority class. It forces the learning algorithm to focus on difficult examples
    \item hand-made rotation of some random samples (axis and angle are both randomized) and shuffling inside a window
\end{itemize}
In both cases the dataset is rebalanced (only the training set) by bringing the three less-represented classes to have around 30\% of the number of samples of the most represented class, which is \textit{standing}.
There are some caveats: activities \textit{lying} and \textit{sitting} are not randomly rotated as this may cause labeling errors due to confusion with similar activities like \textit{standing}, and activities \textit{jumping} and \textit{falling} are not subject to permutation because there may be temporal correlation inside the window (a jump read reversed in time may be confused with a fall).
Adjusted times for both processes are in the fourth column of \tab{tab:act_times}.

Eventually, data is normalized using training set's mean and standard deviation: this is a common procedure in machine learning that might improve accuracy and training.

\section{Learning framework}\label{sec:learning_framework}

We developed networks that works directly processing raw framed signals, and networks that combine automatic feature extraction prior to classification.
For ``plain'' models we have ultimately chosen four alternatives, based on, but not equal to, previous works in the literature, as they differ in combination and number of layers, kernels size and other parameters.
%
For the second part, we explored the combination of CNN and LSTM layers, and we also implemented three alternatives through autoencoders, to be used with SVM, LR and the previously defined networks as classifiers, except one for reasons that will be clear in section~\ref{sssec:ae}.

Combining all these numbers to the six prepared datasets --- with and without normalization, manually augmented, augmented using ADASYN --- for each of the two reference frames, the total number of configurations is $[5+3\times4]\times6\times2=204$\footnote{5 networks, 3 AEs over all but one networks, 6 datasets, 2 reference frames}, excluding SVM and LR tests.

Practically, all the networks are implemented as Keras~\cite{keras2015} models (using TensorFlow~\cite{tensorflow2015} backend).
Training parameters like the optimizer (SGD\footnote{For more information, we refer the reader to \cite{Ruder-SGDoverview}.}, RMSprop~\cite{Graves-RMSprop} or Adam~\cite{Kingma-Adam}), the loss type, the learning rate, its decay and momentum, the number of epochs and batch size are specified and saved separately from models: this modularity allows to perform tests on the same model with different configurations.

\subsection{Plain models}\label{ssec:plain_models}

Here we review the best models we selected.
These are only a small part of all the structures we investigated: much more models can be found in the companion Jupyter notebooks, especially regarding fully connected networks (not included here) and CNNs, since experiments with LSTM and GRU units are mostly related to the number of nodes in the cell and to dropout rate values.
All these models are learned with the Adam optimizer and \textit{categorical crossentropy} loss.
If not differently specified, default values are employed for learning rate ($0.001$), decay rate ($0$) and other parameters.
Investigating hyperparameters in this kind of networks is still a major problem, as pointed out in \cite{Hammerla-DeepConvRec}.
Our values agree with the guidelines defined in the paper.
Dropout modules are used for regularization and to avoid overfitting data.

\subsubsection{1D CNN}

We developed networks based on one or two convolutional layers, plus one or two dense (fully connected) layer followed by \textit{softmax} activation to have class probabilities, with optional \textit{L2} regularization on convolution kernels and dropout after every layer.
As per standard practices, each convolutional layer is followed by batch normalization~\cite{Ioffe-BatchNorm}, ReLU activation and max-pooling.
We additionally included models from surveyed papers~\cite{Chen-SingleAcc,Rueda-CNN}, whose drawbacks have been evidenced in \autoref{sec:related_work}.
A single convolutional layer is not capable to learn from data, overfitting occurs and validation loss always grows.
Adding more than one dense layer at the end leads to longer learning time with no substantial gain in accuracy.
Regularization is not much helpful in CNNs, what keeps the loss stable is dropout, for which the rate is set by trial and error.
The selected model, which has the highest overall and class-wise accuracies, uses two layers, dropout and a final dense layer, and is reported below.

\begin{minipage}[c]{.9\columnwidth}
\vspace*{.5em}
%[float,caption=1D CNN]
\begin{lstlisting}
def <@\textcolor{deeporange}{Conv1D_2C1D_model}@>(input_shape, num_classes=7):
  return Sequential([
    Conv1D(filters=64, kernel_size=5, input_shape=input_shape),
    BatchNormalization(axis=1),
    Activation('relu'),
    Dropout(0.3), # rate
    MaxPooling1D(pool_size=2),
    Conv1D(32, 5),
    BatchNormalization(axis=1),
    Activation('relu'),
    Dropout(0.3),
    MaxPooling1D(2),
    Flatten(),
    Dense(num_classes, activation='softmax')
  ], name='Conv1D-2C1D-do0.3')
\end{lstlisting}
\vspace*{.2em}
\end{minipage}

\subsubsection{2D CNN}

We investigated networks from papers \cite{Bevilacqua-CNN} and \cite{Ha-MultiModalCNN}.
2D convolution can be performed by reshaping data to have the three sensors' signals stacked.
Input data has shape \texttt{(?,9,128,1)} where 128 is due to the framing operation and the last number is the number of channels, only 1 in this case.
While the first network is not brilliant and in our preliminary tests achieves only 86\% accuracy, the second instead is a promising setup: as mentioned in \autoref{sec:related_work}, 95\% accuracy is reached by operating the trick to interpose zero-padding to avoid 2D convolution kernels to sweep more than one signal at a time.
Due to padding, input shape becomes \texttt{(?,18,128,1)}.
As the selected model, it is reported below.

\begin{minipage}[c]{.9\columnwidth}
\vspace*{.2em}
\begin{lstlisting}
def <@\textcolor{deeporange}{Conv2D_Ha_model}@>(input_shape, num_classes=7):
  return Sequential([
    Conv2D(filters=32, kernel_size=(4,4), activation='relu', input_shape=input_shape),
    MaxPooling2D(pool_size=(3,3), strides=(1,1)),
    Conv2D(64, (5,5), activation='relu'),
    MaxPooling2D((3,3),(1,1)),
    Flatten(),
    Dense(128, activation='relu'),
    Dropout(0.5),
    Dense(num_classes, activation='softmax')
  ], name='Conv2D-Ha')
\end{lstlisting}
\vspace*{.2em}
\end{minipage}

\subsubsection{LSTM}
Many tests brought us to set on the model with two LSTM layers, due to improved performance w.r.t. the versions with only one layer. The \textit{bidirectional} configuration is highly dataset-related and very difficult to tune, especially in the number of cells~\cite{Hammerla-DeepConvRec}.
Models with three layers are too slow and do not improve on the selected model, for this dataset.

Different to what is done in \cite{Pienaar-LSTM}, we dropped one LSTM layer and added dropout instead of manually inserting an L2 regularization term in the structure of the LSTM.

\begin{minipage}[c]{.9\columnwidth}
\vspace*{.5em}
\begin{lstlisting}
def <@\textcolor{deeporange}{TwoLSTM_model}@>(input_shape, num_classes=7):
  return Sequential([
    LSTM(512, return_sequences=True, batch_input_shape=input_shape),
    Dropout(0.2),
    LSTM(512, return_sequences=False),
    Dense(num_classes, activation='softmax')
  ], name='TwoLSTM')
\end{lstlisting}
\vspace*{.2em}
\end{minipage}

\subsubsection{GRU}
In this case the one-layer model achieves the same performance of the two-layers one, but in a much longer time (circa $3\times$).
The model with three GRUs instead overfit the data, reaching an accuracy peak slightly lower from the two-layer model, and then performance degrades and loss grows.
In this last case a study on the learning rate decay would be worth, but nonetheless taking into account the LSTM network's results, the fact that the final model already reach nice overall performance, plus the time needed to train such a network leads in this case to drop further experiments on this architecture.

\begin{minipage}[c]{.9\columnwidth}
\vspace*{.5em}
\begin{lstlisting}
def <@\textcolor{deeporange}{TwoGRU_model}@>(input_shape, num_classes=7):
  return Sequential([
    GRU(512, return_sequences=True, input_shape=input_shape),
    Dropout(0.2),
    GRU(512, input_shape=input_shape),
    Dense(num_classes, activation='softmax')
  ], name='TwoGRU')
\end{lstlisting}
\vspace*{.2em}
\end{minipage}

\subsection{Combined models}\label{ssec:combined_models}
We explored the possibility of automatically learn features from raw signals by combining a CNN with a LSTM cell and by means of several autoencoder architectures.

\subsubsection{CNN-LSTM stack}
One ``mixed'' model is obtained by stacking together convolutional and recurrent layers.
The scope is to get the best from both, so to get sort of a ``preprocessing'' step thanks to convolutions, with the goal of learning features that are then handled and classified by LSTMs, thanks to their time-related capabilities.
In this case the input has been reshaped in folds of size \texttt{(?,4,32,9)} to use \texttt{TimeDistributed}, that apply the same \texttt{Dense} operation to every timestep of a 3D tensor\footnote{This is implicit since Keras 2.0 but we prefer to specify it anyway.}.
Moreover, this resulted in an overall speed-up of LSTM layer's training time.

In \cite{Ordonez-CNN-LSTM} the defined network is bigger, featuring two more convolutional layers and one more LSTM, and lack the presence of either dropout and pooling layers. Moreover, the settings about learning and decay rate are different: we use respectively $1\mathrm{e}{\text{--}4}$ and $0$ instead of $1\mathrm{e}{\text{--}2}$ and $0.9$.

\begin{minipage}[c]{.9\columnwidth}
\vspace*{.5em}
\begin{lstlisting}
def <@\textcolor{deeporange}{CNN_LSTM_model}@>(input_shape, num_classes=7):
  return Sequential([
    TimeDistributed(Conv1D(256, 1, activation='relu'), input_shape=input_shape),
    TimeDistributed(Conv1D(256, 3, activation='relu')),
    TimeDistributed(Dropout(0.1)),
    TimeDistributed(MaxPooling1D(2)),
    TimeDistributed(Flatten()),
    LSTM(128),
    Dense(num_classes, activation='softmax')
  ], name='CNN-LSTM')
\end{lstlisting}
\vspace*{.5em}
\end{minipage}

\subsubsection{Autoencoders}\label{sssec:ae}
They should provide an improved representation of the input data by using learned encoded features instead of raw data. %, but we could not see any notable improvement w.r.t. directly using raw signals.
They have been constructed by employing only convolutional layers, only LSTM cells and both of them.
As a first step, data has to be de-framed by a flattening-like operation obtaining sets of shape \texttt{(?\footnote{``?'' is used as wildcard to indicate an unknown number of samples as it depends on the dataset.},1,9)} from \texttt{(?,128,9)}, that is, we process each single measure separately.
We used the defined networks to perform learning also over the original framed dataset and by expanding the last dimension instead of the first one (obtaining a final shape of \texttt{(?,1,128$\times$9)}), but without obtaining good results.

\begin{minipage}[c]{.9\columnwidth}
\vspace*{.2em}
\begin{lstlisting}
def <@\textcolor{deeporange}{CNN_AE_model}@>(input_shape, num_features):
  return Sequential([
    # encoder
    Conv1D(128, 1, activation='relu', padding='same', input_shape=input_shape),
    Conv1D(64,  1, activation='relu', padding='same'),
    # decoder
    Conv1D(64,  1, activation='relu', padding='same'),
    Conv1D(128, 1, activation='relu', padding='same'),
    Conv1D(num_features, 1, activation='softmax')
  ], name='CNN_AE')
\end{lstlisting}
\end{minipage}

\begin{minipage}[c]{.9\columnwidth}
\vspace*{.2em}
\begin{lstlisting}
def <@\textcolor{deeporange}{LSTM_AE_model}@>(input_shape, num_features):
  return Sequential([
    # encoder
    LSTM(128, activation='relu', return_sequences=True, input_shape=input_shape),
    LSTM(64,  activation='relu', return_sequences=True),
    # decoder
    LSTM(128, activation='relu', return_sequences=True),
    TimeDistributed(Dense(num_features, activation='softmax'))
  ], name='LSTM-AE')
\end{lstlisting}
\end{minipage}

\begin{minipage}[c]{.9\columnwidth}
\vspace*{.2em}
\begin{lstlisting}
def <@\textcolor{deeporange}{CNN_LSTM_AE_model}@>(input_shape, num_features):
  return Sequential([
    # encoder
    Conv1D(128, 1, activation='relu', input_shape=input_shape),
    Conv1D(64,  1, activation='relu'),
    # decoder
    LSTM(128, activation='relu', return_sequences=True),
    TimeDistributed(Dense(num_features, activation='softmax'))
  ], name='CNN-LSTM-AE')
\end{lstlisting}
\end{minipage}

After autoencoders have been trained, the best model with respect to accuracy is selected, the decoder side is discarded, the encoded sequences are reshaped back to the original shape and then fed into the CNN-LSTM stack and each of the classification networks defined in \ref{ssec:plain_models}, except the 2D CNN: its particular configuration and padding as a fundamental and logical preprocessing step make it senseless to use this network coupled with autoencoder features.

We further tried also to perform classification through SVMs and LR thanks to a SGD approximation, defined in \textit{scikit-learn}\footnote{\url{https://scikit-learn.org/stable/modules/generated/sklearn.linear_model.SGDClassifier.html}}.
In this case we find out that \textit{alpha}, the multiplier of the regularization term that is also used to compute the initial learning rate, should be in the range $[1\mathrm{e}{\text{--}5},1\mathrm{e}{\text{--}3}]$, with most of the best performance obtained using $1\mathrm{e}{\text{--}4}$.
We combined both SVM and LR with different kind of regularization: L1, L2 and Elastic Net\footnote{A linear combination of L1 and L2 regularization terms.} have been tested.
Overall the possible combinations are $3\times2\times6 = 36$, to be added to the previous 204 tests.

\begin{minipage}[c]{.9\columnwidth}
\vspace*{.5em}
\begin{lstlisting}
sklearn.linear_model.SGDClassifier(loss='hinge', penalty='l2', alpha=0.0001)
# for loss: hinge -> SVM, log -> LR
# for penalty: 'l2', 'l1', 'elasticnet'
\end{lstlisting}
%\vspace*{.5em}
\end{minipage}


\subsection{Metrics}\label{ssec:metrics}

As a multi-class classification problem, to select the best training point (i.e. epoch), beside accuracy, \textit{precision} and \textit{recall} metrics are equally meaningful.
We very briefly recall their definitions, using \tab{tab:class_mat}: the ratio of true positives to the total predicted positives, and the ratio of true positives to all predictions classified in the same class.
%  $TP$ ($TP+FP$) ($TP+FN$)
The \textit{F1-score} is the harmonic average between the two.
Mathematically,
\[ P = \frac{TP}{TP+FP} \qquad R = \frac{TP}{TP+FN} \qquad F1 = 2 \times \frac{P\times R}{P+R} \]

\begin{table}[ht]
    \centering
    \caption{Sample confusion matrix: actual vs predicted class.}
    \label{tab:class_mat}
    \begin{tabular}{c|cc}\toprule
        & Pos & Neg \\\midrule
    Pos & TP  & FN  \\
    Neg & FP  & TN  \\\bottomrule
    \end{tabular}
\end{table}

%\begin{tabular}{c|ccc}\toprule
%          & \multicolumn{3}{c}{Predicted} \\\midrule
%    \multirow{3}{*}{\rotatebox[origin=c]{90}{Actual}} &     & Pos & Neg \\
%          & Pos & TP  & FN  \\
%          & Neg & FP  & TN  \\\bottomrule
%\end{tabular}

Because of the imbalanced dataset, overall accuracy may be misleading since less frequent classes have low impact on the final value, even if for example classification for these samples fails most of the times.
This is enhanced by the fact that test set cannot be augmented.
Because of that, we will focus both on global and per-class metrics, hoping to find a model that maximize each of accuracy values. %, and on loss, that also gives important insights on overfitting.

We introduced these metrics in the training and evaluation phases for all our models, and set up to save a snapshot of the model during training when any of the following quantities (computed on the test set) reach a new maximum:
\begin{itemize}
    \item accuracy: $A$
    \item accuracy over loss: $\mathrm{AoL} = A/L$
    \item sum of acc., precision and recall: $\mathrm{APR} = (A+P+R)$
    \item the previous sum, over loss: $\mathrm{APRoL} = (A+P+R)\ /\ L$
\end{itemize}
Additionally, the model is also saved when at the end of an epoch accuracy is not higher but near current maximum, and loss is lower than the current stored value.

We choose the sum strategy to globally account for each of the metrics.
We understand that having a precise scope for the work, for example like fall detection\footnote{Fall detection may be re-thought as a binary classification problem.}, this reasoning may change a lot, due to possible allowance of false positives and absolute inadmissibility of false negatives.
But this is a general work without a detailed scope, and it is indifferent for our objectives to have higher precision or recall.

We choose the division by the loss to account for the fact that it is also an important value to track during training: more often a model with slightly less accuracy but lower loss is more powerful than a model with slightly higher accuracy and high loss.
This is strictly related to the concept of overfitting.

% !TEX root = template.tex

\section{Results}
\label{sec:results}

\MR{In this section, you should provide the numerical results. You are free to decide the structure of this section. As general rules of thumb, use plots to describe your results, showing, e.g., precision, recall and \mbox{F-measure} as a function of the system (learning) parameters. Present the material in a progressive and logical manner, starting with simple things and adding details and explaining more complex behaviors as you go. Also, do not try to explain / show multiple concepts at a time. Try to \textbf{address one concept at a time}, explain it properly, and only then move to the next one.\\
The best results are obtained by generating the graphs in either \texttt{encapsulated postscript (eps)} or \texttt{pdf} formats. To plot your figures, use the Latex \texttt{includegraphics} command.}


% !TEX root = template.tex

\section{Concluding Remarks}
\label{sec:conclusions}

\red{This section should take max half a page (if you are using for intelligent observations may also be 3/4 of a page :-).}\\

\MR{In many papers, here you find a summary of what done. It is basically an abstract where instead of using the present tense you use the past participle, as you refer to something that you have already developed in the previous sections. While I did it myself in the past, I now find it rather useless.}\\ 

\MR{\textbf{What I would like to see here is:} 
\begin{enumerate}
\item a very short summary of what done, 
\item some (possibly) intelligent observations on the relevance and {\it applicability} of your algorithms / findings, 
\item what is still missing, and can be added in the future to extend your work.\\
\end{enumerate}
The idea is that this section should be {\it useful} and not just a repetition of the abstract (just \mbox{re-phrased} and written using a different tense...).}\\

\red{\textbf{Moreover:} being a project report, I would also like to see a specific paragraph specifying: 
\begin{enumerate}
\item what you have learned, and 
\item any difficulties you may have encountered.
\end{enumerate}}


%\nocite{*} % comment to see only cited papers in bib
\bibliographystyle{IEEEtran}
\bibliography{biblio}
\end{document}
