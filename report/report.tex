% arara: pdflatex
% arara: bibtex
% arara: pdflatex
% arara: pdflatex
\documentclass[journal,10pt,twoside]{IEEEtran}

%\usepackage{algorithm}
%\usepackage{algorithmicx}
%\usepackage{algpseudocode}
\usepackage[utf8]{inputenc}
\usepackage{amsfonts}
\usepackage{amsmath}
\usepackage{amssymb}
\usepackage[T1]{fontenc}
\usepackage{microtype}
%\usepackage{mathtools}
\usepackage[dvipsnames]{xcolor}
\usepackage{graphicx}
\usepackage[tablename=Tab.]{caption}
\usepackage{subcaption}
\usepackage{booktabs}
%\usepackage{import}
\usepackage{multirow}
\usepackage{cite}
\usepackage[export]{adjustbox}
\usepackage{breqn}
%\usepackage{mathrsfs}
\usepackage{acronym}
\usepackage[keeplastbox]{flushend}
\usepackage{setspace}
\usepackage{bm}
\usepackage{stackengine}

\usepackage{smartdiagram}
\usepackage{tikz}
\usesmartdiagramlibrary{additions}

\usepackage{hyperref}
\definecolor{azure}     {rgb}{0,0.5,1}
\definecolor{dkpowder}  {rgb}{0,0.2,0.7}
\definecolor{deepred}   {rgb}{0.7,0,0}
\definecolor{deepblue}  {rgb}{0,0,0.7}
\definecolor{deepgreen} {rgb}{0,0.5,0}
\definecolor{deeporange}{rgb}{0.91, 0.41, 0.17}
\hypersetup{%
    pdfpagemode  = {UseOutlines},
    bookmarksopen,
    pdfstartview = {FitH},
    colorlinks,
    linkcolor = {dkpowder},
    citecolor = {dkpowder},
    urlcolor  = {dkpowder},
}

%\addto\extrasenglish{%
%  \renewcommand{\sectionautorefname}{Section}%
%  \renewcommand{\subsectionautorefname}{Subsection}%
%}

\usepackage{listings}
\lstset{%
backgroundcolor=\color[gray]{.85},
basicstyle=\small\ttfamily,
breaklines = true,
keywordstyle=\color{red!75},
columns=fullflexible,
}%

\lstdefinelanguage{BibTeX}{%
keywords={%
    @article,@book,@collectedbook,@conference,@electronic,@ieeetranbstctl,%
    @inbook,@incollectedbook,@incollection,@injournal,@inproceedings,%
    @manual,@mastersthesis,@misc,@patent,@periodical,@phdthesis,@preamble,%
    @proceedings,@standard,@string,@techreport,@unpublished%
},
comment=[l][\itshape]{@comment},
sensitive=false,
}

% listings settings from classicthesis package by Andr\'{e} Miede
\lstset{language=[LaTeX]Tex,%C++,
    keywordstyle=\color{RoyalBlue},%\bfseries,
    basicstyle=\small\ttfamily,
    %identifierstyle=\color{NavyBlue},
    commentstyle=\color{Green}\ttfamily,
    stringstyle=\rmfamily,
    numbers=none,%left,%
    numberstyle=\scriptsize,%\tiny
    stepnumber=5,
    numbersep=8pt,
    showstringspaces=false,
    breaklines=true,
    frameround=ftff,
    frame=single
    %frame=L
}


\renewcommand{\thetable}{\arabic{table}}
\renewcommand{\thesubtable}{\alph{subtable}}

\DeclareMathOperator*{\argmin}{arg\,min}
\DeclareMathOperator*{\argmax}{arg\,max}

\def\delequal{\mathrel{\ensurestackMath{\stackon[1pt]{=}{\scriptscriptstyle\Delta}}}}

\graphicspath{{images/}}
\setlength{\belowcaptionskip}{0mm}
\setlength{\textfloatsep}{7pt}

\newcommand{\eq}[1]{Eq.~\eqref{#1}}
\newcommand{\fig}[1]{Fig.~\ref{#1}}
\newcommand{\tab}[1]{Tab.~\ref{#1}}
\newcommand{\secref}[1]{Section~\ref{#1}}

\newcommand\MR[1]{\textcolor{blue}{#1}}
\newcommand\red[1]{\textcolor{red}{#1}}
\newcommand\comment[1]{\textcolor{ForestGreen}{[#1]}}

%\renewcommand{\baselinestretch}{0.98}
%\renewcommand{\bottomfraction}{0.8}
%\setlength{\abovecaptionskip}{0pt}
\setlength{\columnsep}{0.15in}

% fix for when no cite is in document but want bib to be shown
\makeatletter
\def\endthebibliography{%
  \def\@noitemerr{\@latex@warning{Empty `thebibliography' environment}}%
  \endlist}
\makeatother

% label colors
\definecolor{label-running} {RGB}{ 31,119,180}
\definecolor{label-walking} {RGB}{255,127, 14}
\definecolor{label-jumping} {RGB}{ 44,160, 44}
\definecolor{label-standing}{RGB}{148,103,189}
\definecolor{label-sitting} {RGB}{140, 86, 75}
\definecolor{label-lying}   {RGB}{127,127,127}
\definecolor{label-falling} {RGB}{188,189, 34}
\definecolor{label-transit} {RGB}{ 23,190,207}

% pass the color to draw the box
\newcommand{\labelbox}[1]{%
    \raisebox{1.6pt}{\fcolorbox{black}{#1}{\rule[2.5pt]{5pt}{0pt}}}
}

% \IEEEoverridecommandlockouts\IEEEpubid{\makebox[\columnwidth]{PUT COPYRIGHT NOTICE HERE \hfill} \hspace{\columnsep}\makebox[\columnwidth]{ }}
\frenchspacing
%\IEEEoverridecommandlockouts

%\title{Going deep into Human Activity Recognition}
%\author{Elia Bonetto, Filippo Rigotto
%\thanks{All authors are with the Department of Information Engineering, University of Padova, Italy. Email: \{bonettoe, rigottof\}@dei.unipd.it}}

\title{Going deep into Human Activity Recognition}
\author{%
    \IEEEauthorblockN{Elia Bonetto and Filippo Rigotto}

    \IEEEauthorblockA{Department of Information Engineering, University of Padova -- Via Gradenigo, 6/b, 35131 Padova, Italy\\ %Email:
        {\tt\{eliabntt94,rigotto.filippo\}@gmail.com}}
}
\markboth{Human Data Analytics, Spring 2019}{Bonetto \& Rigotto: Going deep into Human Activity Recognition}
\IEEEpubid{\raisebox{-1.2pt}{\includegraphics[height=7.5pt]{images/by-sa}} \copyright~2019 The authors. Licensed under \href{https://creativecommons.org/licenses/by-sa/4.0/deed.en}{Creative Commons Attribution -- ShareAlike 4.0}}

\begin{document}

\maketitle
%\thispagestyle{plain} % forced page numbers. TO DELETE before delivery
%\pagestyle{plain} % forced page numbers. TO DELETE before delivery

\begin{abstract}
In latest years, thanks to the increased number of smartphones and wearable devices integrating IMUs, Human Activity Recognition (HAR) has become a key research topic in monitored and assisted living either for medical or tracking reasons.
First attempts provided manual feature crafting, followed by analysis done either with deep neural networks or other approaches like Hidden Markov models.
More recently instead, direct analysis on raw signals has been attempted.
Here we continue this trend by exploring some possible approaches with convolutional and recurrent neural networks and look over automatic feature extraction techniques, such as autoencoders.
Most of the datasets in this field are highly imbalanced and some classes lack of enough data.
To face this, we propose two augmentation techniques for rebalancing.
Finally, we introduce new ways and metrics to select the best learning epoch to address overfitting and get the best learning results overall.
Our tests confirm that augmenting the initial dataset is worth the effort, and we achieve performance that surpass what is declared for it.
Moreover, we discovered that working with raw signals in the sensor reference frame is better than working with their transformation to the body frame.
As for encoded data by means of autoencoders, we could not find any performance improvement: in some cases, worse results are obtained.
\end{abstract}

\begin{IEEEkeywords}
Activity recognition, inertial sensors, machine learning, neural networks, autoencoders, deep classification
\end{IEEEkeywords}

% !TEX root = report.tex

\section{Introduction}
\label{sec:introduction}

%\red{Max len 9 pages. Abstract, intro and related max 2 pages.}\\

%Physical
\IEEEPARstart{P}{hysical} activities recognition, commonly referred as Human Activity Recognition (HAR), has gained momentum as a key research area nowadays.
Tracking and detecting human activities is a relevant task for both trained personnel, e.g. first responders and medical or military services, and for common people: for fitness, for prevention or quick notification of falls, for example in case of elderly people assistance.% and monitoring. fitness reasons

HAR can be performed visually with the aid of many cameras pointed to track people motion but this approach has privacy issues and only works in restricted or indoor areas.
Instead, the use of inertial sensors (IMUs) worn by the users has less drawbacks, it is more robust (for example, it does not rely on the field of view of the camera), cheap and more ubiquitously deployable thanks to the huge spreading of sensor-equipped smartphones and watches~\cite{Wang-survey}.
IMUs are typically composed of accelerometers, gyroscopes and magnetometers that sense and periodically sample linear acceleration, angular velocity and magnetic field variations in the three spatial directions, all within the size of an open hand.
To analyze and classify the data stream offered by these sensors, \textit{features} extraction is needed and can be performed both manually, as in the traditional approach by means for example of the Fourier transform (FFT)~\cite{FrankNadales} or statistical analysis, or by automatic learning through neural network (NN) architectures, a rising trend in the literature~\cite{Wang-survey}.

In this work, we focus on a specific dataset to evaluate the current state of the art, comparing deep networks composed of convolutional and recurrent layers, and we propose some new architectures made as a combination of the previous models or built around the autoencoder (AE) concept to enable automatic feature extraction.
Furthermore, we introduce some balancing techniques for the dataset, trying to achieve better results overall.
Our proposals perform better with respect to the state of the art on the considered dataset.
We also manage to use a lighter version than the original proposal for some of the considered networks.
%\comment{Our proposals are in some cases better than the original architecture in terms of size and efficiency and perform better w.r.t. the state of the art on the considered dataset.}
Implementation is modular, to allow for the use with other datasets, and open-source on \href{https://github.com/eliabntt/deep_into_HAR}{GitHub}, to allow external contributions and future developments.

This work's contributions can be summarized in:
\begin{itemize}
    \item probing usefulness of sensor's frame signal processing
    %\item defining new models as composition of the previous ones, with an eye on taking the best properties from both convolutional and recurrent layers
    \item exploring data augmentation techniques that push forward prediction accuracy
    \item defining and combining (new) models based on established networks that can classify activities from inertial sensor's raw measurements
    \item exploring new metrics to select the best training epoch for models, different from standard setups
    \item making use of the autoencoder architecture to extract features to be fed to deep networks, support vector machines and logistic regression
\end{itemize}

This report is structured as follows.
In \autoref{sec:related_work} we review the current state of the art.
Our system's pipeline is defined in \autoref{sec:processing_pipeline}, preprocessing steps on the original signals' data of the reference dataset are detailed in \autoref{sec:model}.
Analyzed architectures and learning parameters are presented in \autoref{sec:learning_framework}, while in \autoref{sec:results} we outline our results.
\IEEEpubidadjcol

% !TEX root = report.tex

\section{Related work}
\label{sec:related_work}

Activity recognition is a prolific research field and many techniques and algorithms have been used to tackle the subject.
Delving into sensor-based HAR, the trend in the literature until a few years ago, as per~\cite{Lara-Survey-Wearable}, was to manually craft features and then process them by means of Hidden Markov Models (HMM)~\cite{Liano-HMM}, Principal Component Analysis (PCA), Support Vector Machines (SVM), Bayesian Networks (BN)~\cite{Altun-IMU} or Random Forest (RF) ensembles~\cite{Feng-RF}.
Features are extracted using filters, Fourier transforms, moments or other statistical properties (like mean and variance) of the signals~\cite{FrankNadales}.
In this case a heavy preprocessing phase is needed, and these manual features, apart from being most often poorly generalizable, may exclude some important information that can only be extracted by using automatic methods~\cite{Wang-survey}.
Moreover, these methods focus mainly on single-sequence classification, often lacking the study of the temporal correlation between signals since they rely on per-sequence handcrafted features~\cite{Hammerla-DeepConvRec}.
Recent advances in machine learning paved the way to raw signal analysis, automatic feature extraction and classification. In this regard, Wang \textit{et al.}~\cite{Wang-survey} very recently surveyed the literature collecting both model architectures and popular datasets.
%The precedent outdated survey on the topic was redacted by Lara and Labrador~\cite{Lara-Survey-Wearable}, and stops the review of NNs to dense multi-layer perceptrons (MLP), 
%dedicating most of the work on more traditional approaches.
%while Bulling \textit{et al.}~\cite{Bulling-Tutorial} authored a broad tutorial exposing relevant research challenges, introducing a framework to design and evaluate activity recognition systems.

Deep convolutional neural networks (CNN) could learn much more high-level and meaningful features while achieving unparalleled performance: their key advantages are the ability to capture local dependencies and resilience to scale changes~\cite{Zeng-CNN} thanks to their ability to extract hidden information from data.
Temporal 1D convolution is successfully used by Chen and Xue,~\cite{Chen-SingleAcc} who employ a deep CNN with small kernels on data that come only from a single accelerometer.
Even if it is a promising approach, not considering also gyroscope data may lead to underestimated results: accuracy is expected to be lower when dealing with stationary activities.
The authors in~\cite{Lee-1dCNNandRF} do the same, but they consider accelerometer vectors' magnitude instead of raw 3D data to reduce rotational interference. %, and evaluate the network against a random forest classifier.
Moya Rueda \textit{et al.}~\cite{Rueda-CNN} extend the computation considering multiple sensors and organizing data in sliding windows among the set of sensors in parallel.
The authors in~\cite{Zebin} collect and merge data from 5 sensors fixed to different parts of the body, and compare a CNN, a MLP and a SVM: the deep network is both faster and more accurate.
2D convolutions are used by Bevilacqua \textit{et al.}~\cite{Bevilacqua-CNN} to account for spatial and temporal dependencies among signals.
Accelerometer and gyroscope data is stacked and organized in overlapping windows before being fed into a 3-layers convolutional network with small kernels and pooling layers, dropout~\cite{Srivastava-Dropout} and a final 3-layers fully-connected network.
This novel approach leads to even higher accuracy values.
A further improvement by Ha \textit{et al.}~\cite{Ha-MultiModalCNN,Ha-MulAccGyro} consists in separating each sensor's data with padding and adjusting the filter size to not simultaneously perform convolution over different sensors.% and to ensure no cross-interference during learning.

The introduction of recurrent networks (RNN), in particular of Long Short-Term Memory (LSTM)~\cite{Hochreiter-LSTM} and Gated Recurrent Unit (GRU)~\cite{Cho-GRU} cells, allowed to learn temporal sequences dependencies more flawlessly by holding memory of past values.
Thanks to this, recurrent modules are good learners for IMU sensor's data that clearly depends not only from a single value, but from a sequence~\cite{Singh-RNN,Pienaar-LSTM}.
Both LSTM and GRU have been developed to avoid the vanishing/exploding gradient problem of RNNs, and the difference between them is the number of available \textit{update} gates.

In this field most works use LSTM networks~\cite{Guan-LSTM-wearables}.
Simple vanilla approaches like stacking LSTM gates one after another showed an improvement over previous CNNs methods thanks to their relation to time sequences, even if working on plain raw signals. 
An alternative is to use a \textit{bidirectional} network, that is, two LSTMs layers working one on the original sequence (learning from future) and the other one on the reversed version (learning from past), trying to be more robust and essentially doing data augmentation inside the network~\cite{Hammerla-DeepConvRec}.
This approach is the less adopted because it is highly dependant on the number of units of the LSTM cell and so to the input size and the different datasets.
Last, experiments mixing convolutional layers before LSTM(s) classification layers took place~\cite{Ordonez-CNN-LSTM}.
Taking advance from both representation brought by CNNs and temporal dependency by RNNs, results show an improvement with respect to previous methods: a proof that feature extraction techniques before RNNs could be a way to obtain major improvements on HAR and other time-related tasks. 
GRU layers have been less used for this task, probably because they showed lower overall performances\cite{Park-DNN-smartmultisensor,Arifoglu-RNN-HAR}.

% !TEX root = template.tex

\section{Processing Pipeline}
\label{sec:processing_architecture}

\MR{\textbf{On tailoring the paper structure to your needs:} The structure recommended for the previous sections is rather standard and could work for different papers with differing technical content, the structure and the paper content from here on highly depends on the type of paper, possibilities are: mostly based on theoretical analysis, showing experimental design/activity, proposing a new technique and analyzing its performance via experiments or simulation. For the HDA course we deal with machine learning and, in detail, with training and testing neural network architectures to perform some specific inference or classification task. The following structure and comments are specifically addressing this type of technical content.}\\

\MR{\textbf{Why having this section:} With this section, we start the technical description with a {\it high level} introduction of your work (e.g., processing pipeline). Here, you do not have to necessarily go into the technical details of every block/algorithm of your design, this will be done later as the paper develops. What I would like to see here is a high level description of the approach, i.e., which processing blocks you used, what they do (in words) and how these were concatenated, etc. This section should introduce the reader to your design, explain the different parts/blocks, how they interact and why. You should not delve into technical details for each block, but you should rather explain the big picture.} \red{Besides a well written explanation, I often use a nice diagram containing the various blocks and detailing their interrelations.}

\section{Signals and Features}
\label{sec:model}

\MR{Being a machine learning paper, I would put here a section describing the signals you have been working on. If possible, you should describe, in order, 1) the measurement setup, 2) how the signals were \mbox{pre-processed} (to remove noise, artifacts, fill gaps or represent them through a constant sampling rate, etc.). After this, you should describe how {\it feature vectors} were obtained from the \mbox{pre-processed} signals. If signals are {\it time series} this also implies stating the segmentation / windowing strategy that was adopted, to then describe how you obtained a feature vector for each time window. Also, if you also experiment with previous feature extraction approaches, you may want to list them as well, in addition to (and before) your own (possibly new) proposal.}

\MR{Last but not least, this section should also contain information on how you have split the dataset into training and test \mbox{sub-sets}. This will be then recalled within the ``Results'' section.}

\section{Learning Framework}
\label{sec:learning_framework}

\MR{Here you finally describe the learning strategy / algorithm that you conceived and used to solve the problem at stake. A good diagram to exemplify how learning is carried out is often very useful. In this section, you should describe the learning model, its parameters, any optimization over a given parameter set, etc. You can organize this section in \mbox{sub-sections}. You are free to choose the most appropriate structure.}

\red{Note that the diagram that you put here differs from that of Section~\ref{sec:processing_architecture} as here you show the details of how your learning framework, or the core of it, is built. In Section~\ref{sec:processing_architecture} you instead provide a high-level description of the involved processing blocks, i.e., you describe the {\it processing flow} and the rationale behind it.} \\

\MR{\textbf{On the math typesetting:} there are many Latex tricks that you should use to produce a high quality technical essay. A few are listed below, in random order:
\begin{itemize}
\item \textbf{Vectors and matrices:} $x$ is a scalar, whereas $\bm{x}$ (in bold) is a vector, and $\bm{X}$ is a matrix with elements $\bm{X} = [x_{ij}]$. For bold symbols use the \texttt{\textbackslash bm} Latex command, e.g., \texttt{\textbackslash bm(x)}.
\item \textbf{Operators:} such as $\max$, $\min$, $\arg\!\max$, $\arg\!\min$ and special functions such as $\log(\cdot)$, $\exp(\cdot)$, $\sin(\cdot)$, $\cos(\cdot)$ are obtained through specific latex commands \texttt{\textbackslash min}, \texttt{\textbackslash max}, \texttt{\textbackslash arg\textbackslash!min}, \texttt{\textbackslash arg\textbackslash!max}, \texttt{\textbackslash log}, \texttt{\textbackslash exp}, \texttt{\textbackslash sin}, \texttt{\textbackslash cos}, etc. Use them! $log$, $exp$, $min$, $sin$, $cos$, etc., look ugly.
\item \textbf{Sets} can be represented through calligraphic fonts, e.g., $\mathcal{S}$, $\mathcal{F}$, $\mathcal{B}$, etc., obtained using the Latex command \texttt{\textbackslash mathcal{\{S\}}}, etc.
\item \textbf{Equations:} for a single equation use the \texttt{equation} Latex environment. Example:
\begin{equation}
\label{eq:sigmoid}
\sigma(x) = \frac{1}{1+e^{-x}}.
\end{equation}
Now, using round brackets $($ and $)$ we get
\begin{equation}
\label{eq:sigmoid_comma}
\sigma(x) = (\frac{1}{1+e^{-x}}),
\end{equation}
but this looks ugly, you should use ``\texttt{\textbackslash left (}'' for ``$($'' and ``\texttt{\textbackslash right )}'' for ``$)$'', obtaining
\begin{equation}
\sigma(x) = \left ( \frac{1}{1+e^{-x}} \right ).
\end{equation}
\item \textbf{Punctuation:} Displayed equations are usually considered to be part of the preceding sentence and, in turn, they will get the very same punctuation as if they were inline with the text. If the sentence ends in a displayed equation, the equation gets a period ``.'' right after it, see Eq.~(\ref{eq:sigmoid}). iI the equation is instead part of a running sentence, which is continued after it, then the equation may be ended by a ``,'' as in Eq.~(\ref{eq:sigmoid_comma}). Use the standard grammar rules and your good sense of flow to assess how equations should be punctuated, I usually read through as if they were plain text.
\end{itemize}}

% !TEX root = report.tex

\section{Results}\label{sec:results}

The 12 generated datasets, as described in \autoref{sec:model}, are:
%As a summary of the 12 generated datasets, they are:
\begin{itemize}
    \item w.r.t. two reference frames (sensor and body)
    \item with and without normalization
    \item optionally augmented manually or with ADASYN
\end{itemize}
The results are explained in a top-down fashion to progressively prune out combinations.
Subsequent considerations still apply to previously excluded parts.
The complete set of results can be generated using the notebooks.
Note that for Keras models and scikit-learn's SGD implementations accuracy equals weighted recall, so we will use this as comparison metric.

%We want to put evidence in the fact that performance differences in some cases are minimal and can be linked either to the stochastic nature of the methods and to the fact that our models perform very well on the most represented classes (around 98/100\%).
%The complete set of results can be generated using the publicly available companion notebook.

\subsection{Reference frame}
Comparing the same networks, trained over the same type of dataset, we can note how performance are superior with sensor-referenced data, instead of using data referenced to the body frame, and it is easy to see how even plain global accuracy is lower on such datasets, an example is in \tab{tab:body_sensor}.
This is a surprising aspect because the coordinate transformation should act as a stabilizer to equalize measurements and generalize them as noted in~\cite{Liano-HMM}.

\begin{table}[ht]
\centering
\caption{Best accuracy value (\%) on some datasets, body (\texttt{B*}) and sensor (\texttt{S*}) reference frame. \textit{Mixed} is CNN-LSTM, due to space constraints.}
\label{tab:body_sensor}
\begin{tabular}{lrrrrr}
\toprule
net &  Conv1D-2C1D-do0.3 &  Conv2D-Ha &  TwoLSTM &  TwoGRU &  CNN-LSTM \\
ds   &                    &            &          &         &           \\
\midrule
BADA &             98.062 &     95.528 &   98.085 &  98.085 &    98.447 \\
BNOR &             97.933 &     95.166 &   98.120 &  98.143 &    98.435 \\
SADA &             99.194 &     97.898 &   99.194 &  99.159 &    99.451 \\
SNOR &             99.159 &     98.073 &   99.159 &  99.113 &    99.358 \\
\bottomrule
\end{tabular}

\end{table}

\subsection{Normalization}
As common practice suggest we performed normalization on datasets using training set's mean and standard deviation.
In general, by looking at global accuracy, there is no strict rule to say that using it is better than leaving data as is.
The results are so near that could even be linked to stochasticity of trainings and to highly imbalanced classes.
From \tab{tab:norm-aug} we can note how there are cases like CNN-LSTM and SADA(n) where normalized dataset goes better, and cases like TwoGRU and SAHC(n) where it is the opposite.
So we cannot come up with a definitive answer for this normalization question.

\begin{table}[ht]
\centering
\caption{Best accuracy value (\%) for sensor-referenced datasets: ADASYN-augmented (\texttt{SADA}), manually augmented (\texttt{SAHC}), normalized but not augmented (\texttt{SFRA}). Not normalized versions of these three datasets are \texttt{SADAn}, \texttt{SAHCn} and \texttt{SFRAn}.}
\label{tab:norm-aug}
\begin{tabular}{c|ccccc}\toprule
Dataset & Conv1D            & Conv2D            & 2LSTM             &  2GRU             & Mixed \\\midrule
SADA    & \textbf{99.194}   & 97.898            & 99.194            & 99.159            & \textbf{99.451} \\
SADAn   & 99.066            & 98.003            & \textbf{99.241}   & \textbf{99.288}   & 99.183 \\
SAHC    & 97.957            & \textbf{98.284}   & 99.206            & 97.840            & 99.416 \\
SAHCn   & 99.043            & 98.260            & 99.229            & 99.253            & 99.183 \\
SFRA    & 99.159            & 98.073            & 99.159            & 99.113            & 99.358 \\
SFRAn   & 98.984            & 98.272            & 98.564            & 99.206            & 99.113 \\\bottomrule
\end{tabular}
\end{table}

% not related here but goes to next page
\begin{table*}[ht]
\centering
\caption{Saving epoch, per-class accuracy, global accuracy, weighted and standard precision and recall (\%), for the two most prominent models.
Each value is reported according to the four metrics presented in section~\ref{ssec:metrics} (top to bottom, same order).}
\label{tab:metrics}
\begin{tabular}{lc|ccccccc|cccc}\toprule
Model & E &\labelbox{label-running} & \labelbox{label-walking} & \labelbox{label-jumping} & \labelbox{label-standing} & \labelbox{label-sitting} & \labelbox{label-lying} & \labelbox{label-falling} & Acc. & W Prec. & Prec. & Recall \\\midrule
\multirow{2}{*}{2GRU}     & 101 & 100.000 & 99.014 & 93.897 & 99.588 & 99.637 & 99.874 & 87.273 & 99.288 & 99.284 & 99.293 & 99.282 \\
                          &  84 & 100.000 & 99.113 & 90.141 & 99.706 & 99.577 & 99.748 & 90.909 & 99.264 & 99.262 & 99.293 & 99.247 \\
\multirow{2}{*}{(SADAn)}  & 101 & 100.000 & 99.014 & 93.897 & 99.588 & 99.637 & 99.874 & 87.273 & 99.288 & 99.284 & 99.293 & 99.282 \\
                          &  84 & 100.000 & 99.113 & 90.141 & 99.706 & 99.577 & 99.748 & 90.909 & 99.264 & 99.262 & 99.293 & 99.247 \\\midrule
\multirow{2}{*}{Mixed}    &  27 & 100.000 & 99.211 & 92.488 & 99.794 & 99.758 & 99.874 & 94.545 & 99.451 & 99.454 & 99.464 & 99.452 \\
                          &  21 & 100.000 & 99.113 & 92.488 & 99.676 & 99.819 & 99.874 & 92.727 & 99.381 & 99.382 & 99.394 & 99.382 \\
\multirow{2}{*}{(SADA)}   &  27 & 100.000 & 99.211 & 92.488 & 99.794 & 99.758 & 99.874 & 94.545 & 99.451 & 99.454 & 99.464 & 99.452 \\
                          &  21 & 100.000 & 99.113 & 92.488 & 99.676 & 99.819 & 99.874 & 92.727 & 99.381 & 99.382 & 99.394 & 99.382 \\\bottomrule
\end{tabular}

\end{table*}

\subsection{Augmentation}
Evaluation is problematic also in this section: from \tab{tab:norm-aug} we can note how augmented datasets do not \textit{always} perform better than non-augmented ones.
What is noted instead is that \textit{at least} one of the two augmented versions performs better than ``plain'' ones.
%
%\comment{As can be seen in \tab{tab:norm-aug}, also here w.r.t. augmentation there are a bit of problems since augmented datasets do not \textit{always} perform better than not augmented ones.
%What we can surely see instead is that \textit{at least} one of the two augmented versions performs better than plain ones.
%Moreover ADASYN augmentation achieve better results wrt manual one in all networks but Conv2, taking in consideration both normalized and not normalized versions of the ds.}
%\comment{This behaviour could was expected because augmentation is performed to enhance less represented classes and as such they do not have great influence on a global metric. A proof of that can be seen in \fig{fig:cm-sahc-conv2d} and \fig{fig:cm-snor-conv2d} where is clear that \textit{jumping} and \textit{falling} reaches way better per-class accuracies.}
This behavior was expected, because augmentation is performed to enhance less represented classes and as such they do not have great influence on a global metric.
A proof can be seen in \fig{fig:cm-sahc-conv2d} and \fig{fig:cm-snor-conv2d} where clearly \textit{jumping} and \textit{falling} reaches way better per-class accuracies.
Moreover, ADASYN achieve better results w.r.t. manual augmentation in all networks but Conv2D, considering both normalized and non-normalized versions.
This also proves augmentation is useful and worth.
Paired with previous results, here are the best models overall: TwoGRU over SADAn, CNN-LSTM over SADA.

\begin{figure}[ht]
\hspace*{+0.15cm}
    \centering
    \includegraphics[width=1.05\columnwidth]{images/plot-cm-SAHC_Conv2D.png}
    \caption{Confusion matrix of Conv2D model, SAHC dataset.}
    \label{fig:cm-sahc-conv2d}
\end{figure}

\begin{figure}[ht]
\hspace*{+0.15cm}
    \centering
    \includegraphics[width=1.05\columnwidth]{images/plot-cm-SNOR_Conv2D.png}
    \caption{Confusion matrix of Conv2D model, SFRA dataset.}% because snor == sfra and sfra == sfran
    \label{fig:cm-snor-conv2d}
\end{figure}

\subsection{Models and metrics}

By looking at \tab{tab:metrics}, we notice how:
\begin{itemize}
\begin{samepage}
    \item defined metrics may couple together: this happens when the model is saved at the same epoch according to two different metrics. Most of the times $\mathrm{A}$ couples with $\mathrm{APR}$ while $\mathrm{AoL}$ couples with $\mathrm{APRoL}$, but there may be exceptions like CNN-LSTM on SAHCn, where $\mathrm{A}$ differs from $\mathrm{APR}$.
    \item generally, accuracy $\mathrm{A}$ (weighted recall) gives best results.
    \item the best model is CNN-LSTM.
\end{samepage}
\end{itemize}

In general, our models achieve high accuracy within a small number of epochs ranging from a few to a hundred or so.
2D CNN and CNN-LSTM exhibit slight overfitting (see \fig{fig:acc-loss-sada-mixed}), that could be tackled with additional regularization or learning rate's decay.
Anyway, models are saved before this happens.

\begin{figure}[ht]
    \centering
    \includegraphics[width=\columnwidth]{images/plot-acc-loss-SADA_CNN-LSTM.png}
    \caption{CNN-LSTM, SADA dataset: training accuracy and loss.}
    \label{fig:acc-loss-sada-mixed}
\end{figure}

Only comparing the two best models in \tab{tab:norm-aug}, CNN-LSTM performs slightly better, even if the 0.163\% gained (from 99.288\% to 99.451\%) is a stunning 22.9\% of the remaining available accuracy.
Taking the maximum and minimum values, we span from 97.840\% to 99.451\%, covering about 74.6\% of the remaining achievable accuracy.

\begin{table*}
\centering
\caption{Comparison between our best networks and datasets combination and original results from \cite{FrankNadales} based on Dynamic Bayesian Networks: precision (first) and recall (second) values for each class. Best results in bold (virtually rounded).}
\label{tab:soa}
%\begin{tabular}{l|cllllllllllllll}\toprule
%& \multicolumn{2}{c}{\textbf{RUNNING}} & \multicolumn{2}{c}{WALKING} & \multicolumn{2}{c}{\textbf{JUMPING}} & \multicolumn{2}{c}{STANDING} & \multicolumn{2}{c}{SITTING} & \multicolumn{2}{c}{\textbf{LYING}} & \multicolumn{2}{c}{\textbf{FALLING}} \\ \midrule
%
%\begin{tabular}{l|ccccccc}\toprule
%Model  & \labelbox{label-running} & \labelbox{label-walking} & \labelbox{label-jumping} & \labelbox{label-standing} & \labelbox{label-sitting} & \labelbox{label-lying} & \labelbox{label-falling} \\\midrule
%Conv1D & 97.912~~99.764       & \textbf{98.869}~~99.113 & \textbf{98.990}~~92.019          & 99.325~~\textbf{99.647}          & \textbf{99.879}~~99.698 & 99.369~~\textbf{99.369}       & \textbf{89.796}~~80 \\
%Conv2D & 96.948~~97.636       & \textbf{98.277}~~98.374 & 90.196~~86.385                   & 98.969~~\textbf{98.940}          & \textbf{98.315}~~98.731 & 99.246~~\textbf{99.622}       & 80~~72.727 \\
%2LSTM  & 98.601~~\textbf{100} & \textbf{99.407}~~99.162 & \textbf{98.030}~~\textbf{93.427} & 99.149~~\textbf{99.499}          & \textbf{99.697}~~99.517 & 99.873~~\textbf{99.369}       & \textbf{86.441}~~92.727 \\
%2GRU   & 99.063~~\textbf{100} & \textbf{99.259}~~99.014 & \textbf{95.694}~~\textbf{93.897} & 99.441~~\textbf{99.588}          & \textbf{99.517}~~99.637 & 99.497~~\textbf{99.874}       & \textbf{96}~~87.272 \\
%Mixed  & 97.465~~\textbf{100} & \textbf{99.407}~~99.211 & \textbf{98.995}~~92.488          & \textbf{99.559}~~\textbf{99.794} & \textbf{99.819}~~99.758 & \textbf{100}~~\textbf{99.874} & \textbf{92.857}~~94.545 \\
%DLR    & \textbf{100}~~93     & 98~~\textbf{100}        & 93~~93                           & \textbf{100}~~98                 & 97~~100                 & 100~~98                       & 80~~100 \\ \bottomrule
%\end{tabular}
%
\begin{tabular}{l|ccccccc}\toprule
Model  & \labelbox{label-running} running & \labelbox{label-walking} walking & \labelbox{label-jumping} jumping & \labelbox{label-standing} standing & \labelbox{label-sitting} sitting & \labelbox{label-lying} lying & \labelbox{label-falling} falling \\\midrule
Conv1D\footnotesize{-2C1D} & 97.91~~99.76 & \textbf{98.87}~~99.11 & \textbf{98.99}~~92.02 & 99.33~~\textbf{99.65}         & \textbf{99.88}~~99.70 & 99.37~~\textbf{99.37}        & \textbf{89.80}~~80.00 \\
Conv2D-Ha & 96.95~~97.64        & \textbf{98.28}~~98.37 & 90.20~~86.39                   & 98.97~~\textbf{98.94}          & \textbf{98.32}~~98.73 & 99.25~~\textbf{99.62}        & 80.00~~72.73 \\
TwoLSTM   & 98.60~~\textbf{100} & \textbf{99.41}~~99.16 & \textbf{98.03}~~\textbf{93.43} & 99.15~~\textbf{99.50}          & \textbf{99.70}~~99.52 & \textbf{99.87}~~\textbf{99.37}        & \textbf{86.44}~~92.73 \\
TwoGRU    & 99.06~~\textbf{100} & \textbf{99.26}~~99.01 & \textbf{95.69}~~\textbf{93.90} & 99.44~~\textbf{99.59}          & \textbf{99.52}~~99.64 & \textbf{99.50}~~\textbf{99.87}        & \textbf{96.00}~~87.27 \\
CNN-LSTM  & 97.47~~\textbf{100} & \textbf{99.41}~~99.21 & \textbf{98.99}~~92.49          & \textbf{99.56}~~\textbf{99.79} & \textbf{99.82}~~99.76 & \textbf{100}~~\textbf{99.87} & \textbf{92.86}~~94.55 \\
BN \cite{FrankNadales}  & \textbf{100}~~93    & 98~~\textbf{100}      & 93~~93                         & \textbf{100}~~98               & 97~~100               & 100~~98                      & 80~~100 \\ \bottomrule
\end{tabular}

\end{table*}

As per \tab{tab:soa}, comparing results with the work done in \cite{FrankNadales}, which is based on Dynamic unrestricted Bayesian Networks, we achieve better precision and recall in most of the classes, especially in the augmented ones, with almost all our best configurations.

Finally, in general TwoLSTM, TwoGRU and CNN-LSTM are the models that perform, regardless of normalization and augmentation, \textit{always} better than Conv2D-Ha and \textit{almost} constantly better than Conv1D-2C1D.

As for timings, we notice in \tab{tab:timings} that recurrent networks are much slower with respect to convolutional models, yielding only a low overall performance increase.
Notably, thanks to pooling and convolutional layers, the combined model actually achieves the best training speed overall, letting us conclude along with previous considerations that it is the best in our pool of analyzed networks.

\begin{table}[ht]
\centering
\caption{Mean sample processing time and epoch duration, \texttt{SAHC} dataset, 27611 samples. Benchmark on dual-core 2.30GHz CPU and NVIDIA\textregistered{} T4.}
% shape \texttt{(27611,128,9)}
\label{tab:timings}
\begin{tabular}{c|cccccc}\toprule
       & Conv1D & Conv2D  & 2LSTM & 2GRU  & Mixed \\\midrule
Sample & 235us  & 219us   & 3ms   & 2ms   & \textbf{175us} \\
Epoch  & 6.6s    & 6.0s   & 75s   & 53.8s & \textbf{4.8s}  \\\bottomrule
\end{tabular}
\end{table}

\subsection{Autoencoders}
Training of autoencoders is good and fast: as reported in \tab{tab:ae_train}, they reach 99\% accuracy on test set with a loss of 0.25 in less than five epochs.

\begin{table}[ht]
\centering
\caption{Accuracy and loss of trained AEs, for some datasets.}
\label{tab:ae_train}
\begin{tabular}{lrrrrrr}
\toprule
{} & \multicolumn{3}{l}{acc} & \multicolumn{3}{l}{loss} \\
net & CNN-LSTM-AE & CNN\_AE & LSTM-AE & CNN-LSTM-AE & CNN\_AE & LSTM-AE \\
ds   &             &        &         &             &        &         \\
\midrule
BAHC &       0.996 &  0.996 &   0.994 &       0.250 &  0.250 &   0.250 \\
BNOR &       0.993 &  0.994 &   0.993 &       0.812 &  0.812 &   0.812 \\
SADA &       0.997 &  0.996 &   0.955 &       0.581 &  0.581 &   0.586 \\
SAHC &       0.995 &  0.996 &   0.995 &       0.385 &  0.385 &   0.385 \\
SFRA &       0.998 &  0.986 &   0.986 &      12.099 & 12.101 &  12.101 \\
SNOR &       0.996 &  0.997 &   0.994 &       0.793 &  0.793 &   0.793 \\
\bottomrule
\end{tabular}

\end{table}

Instead, application of encoded data to the networks have not brought the expected improvements in any datasets or methods we used.
In general we noted comparable, and in some cases slightly lower, performance than the same networks trained on the original dataset. A few examples can be seen in \tab{tab:ae_comp}.

\begin{table}[ht]
\centering
\caption{Comparison between models trained on encoded data and on the original data.}
\label{tab:ae_comp}
\begin{tabular}{c|cccccc} \toprule
Model  & Dataset   & AE        & DS Acc. & AE Acc. \\\midrule
Conv1D & SAHC      & LSTM      & 99.416  & 98.751  \\
TwoGRU & BFRA      & CNN       & 98.085  & 96.298  \\
Mixed  & SADA      & CNN-LSTM  & 99.451  & 98.459  \\\bottomrule
\end{tabular}

\end{table}

Apart from being time costly to permute all of them, applications of SVM and LR in combination with L1, L2 and Elastic Net regularization terms do not provide good results: models tend to well learn long and stable activities, while being worse with \textit{running}, \textit{jumping} or \textit{falling} (see \fig{fig:cm-svm-sahc-mixed}).
We reached a top value of 90\% global test accuracy on a specific combination of AE and LR model.
Furthermore, there is no clear linking between performance and LR or SVM or with respect to regularization terms, which makes essential to loop all combinations to see the best one.

\begin{figure}
    \centering
    \includegraphics[width=\columnwidth]{images/plot-cm-SVM-L2-SAHC_Mixed.png}
    \caption{SVM with L2 regularization, CNN-LSTM encoding, SAHC dataset.}
    \label{fig:cm-svm-sahc-mixed}
\end{figure}

The pattern of results obtained over each dataset and network is in accordance with our previous findings: these methods cannot compete with the previously defined ones, and in our implementation it is better to process raw signals than to build methods on encoded features.

% !TEX root = template.tex

\section{Concluding Remarks}
\label{sec:conclusions}

\red{This section should take max half a page (if you are using for intelligent observations may also be 3/4 of a page :-).}\\

\MR{In many papers, here you find a summary of what done. It is basically an abstract where instead of using the present tense you use the past participle, as you refer to something that you have already developed in the previous sections. While I did it myself in the past, I now find it rather useless.}\\ 

\MR{\textbf{What I would like to see here is:} 
\begin{enumerate}
\item a very short summary of what done, 
\item some (possibly) intelligent observations on the relevance and {\it applicability} of your algorithms / findings, 
\item what is still missing, and can be added in the future to extend your work.\\
\end{enumerate}
The idea is that this section should be {\it useful} and not just a repetition of the abstract (just \mbox{re-phrased} and written using a different tense...).}\\

\red{\textbf{Moreover:} being a project report, I would also like to see a specific paragraph specifying: 
\begin{enumerate}
\item what you have learned, and 
\item any difficulties you may have encountered.
\end{enumerate}}


%\nocite{*} % comment to see only cited papers in bib
\bibliographystyle{IEEEtran}
\bibliography{biblio}
\end{document}
